\documentclass[a4paper,12pt]{article}
\usepackage{amsmath, amssymb}

\begin{document}
\title{Assignment 2}
\author{Nitin Gangahar}
\date{February 2013}
\maketitle

\section*{2.17}
	Rearranging coefficients we have
	\begin{equation}
		A_N = A_{N-1}(1-\frac{4}{N}-\frac{2}{N}) + 2
	\end{equation}
	Multiplying both sides by N
	\begin{equation}
		N A_N = A_{N-1} (N-6) + 2N
	\end{equation}

	The factor to divide would be 
	\centerline{\fbox{
		$\frac{6.5.4\ldots1}{N.(N-1)\ldots(N-5)}$
	}}

	Assume $N>6$ and applying the theorem given in Chapter 2, we get
	\begin{equation}
		A_N = 2 + \sum_{j>=6}^{j<N} 2 \times \frac{j-5}{j+1} \times \frac{j-4}{j+2} \times \ldots \times \frac{N-6}{N}
	\end{equation}

	This can be changed to
	\begin{equation*}
		A_N = 2 + 2 \sum_{j>=6}^{j<N} \frac{j! \times (n-6)!}{n! \times (j-6)!}
	\end{equation*}

	Multiplying and dividing by $6!$ inside the sum gives
	\begin{equation}
		A_N = 2 [ 1 + \sum_{j>=6}^{j<N} \frac{\binom{j}{6}}{\binom{N}{6}} ]
	\end{equation}

	Looking up wikipedia for binomial coefficients gives us the following equation which can be used
	\begin{equation}
		\sum_{m=0}^{n} \binom{m}{k} = \binom{n+1}{k+1}
	\end{equation}

	This gives
	\begin{equation}
		A_N = 2 \frac{\binom{N+1}{7}}{\binom{N}{6}}
	\end{equation}

	Which gives the final closed recurrence
	\centerline{
		\fbox{
			$A_N = 2\frac{N+1}{7}$
		}
	}
	We can solve the initial cases $N<=6$ manually
	$\square$
\section*{2.69}
	Solved in the forums. The solution I was getting was not specific enough. 
\section*{3.20}
	We have the following
	\begin{equation}
		a_n = 3a_{n-1} - 3a_{n-2} + a_{n-3} n>2 \text{and} a_0 = a_1 = 0 a_2 = 1
	\end{equation}

	First set of initial conditions and using usual method of GF gives
	\centerline{\fbox{$A_N = \binom{N}{2}$}}
	for $ N >= 2 $
	Second condition when used along with backward convolution can be solved and the result varies from the previous result. Hence, the initial conditions make a huge difference. 
	\centerline{\fbox{$a_n = \frac{n(3-n)}{2}$}}
	$\square$
\section*{3.28}
	By expanding with taylor series and differentiating both sides and also using the following property
	\fbox{$\frac{d(a^x)}{dx} = (ln a)a^x$}	we get
	\begin{equation*}
		\text{Required coefficient}(R) = \binom{\alpha+k-1}{k} \left( \frac{1}{\alpha} + \frac{1}{\alpha + 1} + \ldots + \frac{1}{\alpha + k -1} \right)
	\end{equation*}
	and substituting $\alpha = \frac{1}{2}$
	\begin{equation*}
		R = \frac{(k-\frac{1}{2})!}{k!} \left( 2 + \frac{2}{3} + \ldots + \frac{2}{2k-1} \right)
	\end{equation*}

	The term $\frac{k-\frac{1}{2}!}{k!}$ can be solved to give $\frac{1}{4^n}\binom{2n}{n}$
	Also the series inside can be evaluated to give the final answer for $R$ as

	\fbox{
		$
			R = \frac{1}{4^n} \binom{2n}{n} (2H_{2n} - H_n)
		$
	}

\end{document} 




