\documentclass[a4paper,12pt]{article}
\usepackage{amsmath}
\begin{document}
\title{Assignment 1}
\author{Nitin Gangahar}
\date{February 2013}
\maketitle

\section*{Question 1}

\begin{equation}
	A_N = 1 + \frac{2}{N} \sum_{j=1}^{N} A_{j-1}  for N > 0
\end{equation}

Multiplying both sides by N gives
\begin{equation}
	NA_N = N + 2 \sum_{j=1}^{N} A_{j-1} for N > 0
\end{equation}

This holds for N-1 as
\begin{equation}
	(N-1)A_{N-1} = (N-1) + 2\sum_{j=1}^{N-1} A_{j-1} for N > 1 
\end{equation}

Subtracting the last two
\begin{equation}
	NA_N = 1 + (N+1)A_{N-1}
\end{equation}

Dividing both sides by $N(N+1)$ gives
\begin{equation}
	\frac{A_N}{N+1} = \frac{1}{N(N+1)} + \frac{A_{N-1}}{N}
\end{equation}

Iterating
\begin{equation}
	\frac{A_N}{N+1} = \sum_{k=2}^{N}\frac{1}{k(k+1)} + \frac{A_1}{2}
\end{equation}

Assuming $A_0 = 0$ as mentioned in the errata and solving the equation gives

\centerline{\fbox{$A_N = N$}}

\newpage
\section*{Question 2}
Let's assume our pivot is the $x+1^{th}$ smallest element in the array of N numbers after it is partitioned. Hence, there will be x elements to the left of the element once it gets partitioned. 

Hence, we need to calculate the following - 
\begin{equation} \label{eq:q2main}
	\frac{1}{N} \sum_{x=0}^{N-1} E(\textbf{number of elements greater than the pivot in the first x elements}) 
\end{equation}

This is so because only the number of elements greater than the pivot in the first x elements will need to be exchanged. 

The quantity in the summation can be calculated using linearity of expectation.Say, there are variables $X_1 \ldots X_x$ where $X_i = 1i$ if $i^{th}$ element is greater than pivot else 0

Now, by linearity of expectation 
\begin{equation}
	\begin{split}
	E(\textbf{Number of elements greater than pivot in the first x elements}) \\
		= E(X_1+X_2+ \ldots + X_x)
	\end{split}
\end{equation}
But, notice that $E(X_i) = Pr(X_i)$ and 

\begin{equation}
	P(X_i = 1) = 1 - \frac{x}{N-1}
\end{equation}

Hence, using the above observations and plugging it into \eqref{eq:q2main}, we get the required result.

\centerline{\fbox{$\text{ Number of exchanges before partitioning } = \frac{(N-2)}{6}$}}

\newpage
\section*{Question 3}
We solve this with exactly the same method for solving the usual Quicksort recurrence. Notice that on the iteration step we can't go all the way to $C_1$ but have to stop at $C_M$ and this gives us the following recurrence. 

\begin{equation}
	\frac{C_N}{N+1} = 2 \sum_{k=M+2}{N+1} \frac{1}{k} + \frac{C_M}{M+1}
\end{equation}

Replacing the value of $C_M$ gives us
\fbox{ $C_N = 2 \sum_{k=M+2}^{N+1} \frac{1}{k} + \frac{M(M-1)}{4(M+1)}$ } 

\section*{Question 4}
Using the approximation to $H_N$ gives us the following value for $f(M)$

\begin{equation}
	f(M) = \frac{M(M-1)}{4(M+1)} - 2ln(M+2)
\end{equation}

Throwing away $\frac{M-1}{M+1}$ gives us the minimum to occur at $M = 6$

Pretty cool huh! 
\end{document} 




