\documentclass[a4paper,12pt]{article}

\begin{document}
\title{Assignment 1}
\author{Nitin Gangahar}
\date{February 2013}
\maketitle

\section*{Question 1}

\begin{equation}
	A_N = 1 + \frac{2}{N} \sum_{j=1}^{N} A_{j-1}  for N > 0
\end{equation}

Multiplying both sides by N gives
\begin{equation}
	NA_N = N + 2 \sum_{j=1}^{N} A_{j-1} for N > 0
\end{equation}

This holds for N-1 as
\begin{equation}
	(N-1)A_{N-1} = (N-1) + 2\sum_{j=1}^{N-1} A_{j-1} for N > 1 
\end{equation}

Subtracting the last two
\begin{equation}
	NA_N = 1 + (N+1)A_{N-1}
\end{equation}

Dividing both sides by $N(N+1)$ gives
\begin{equation}
	\frac{A_N}{N+1} = \frac{1}{N(N+1)} + \frac{A_{N-1}}{N}
\end{equation}

Iterating
\begin{equation}
	\frac{A_N}{N+1} = \sum_{k=2}^{N}\frac{1}{k(k+1)} + \frac{A_1}{2}
\end{equation}

Assuming $A_0 = 0$ as mentioned in the errata and solving the equation gives

\centerline{\fbox{$A_N = N$}}

\newpage
\section*{Question 2}
Let's assume our pivot is the $x+1^{th}$ smallest element in the array of N numbers after it is partitioned. Hence, there will be x-1
\end{document}
